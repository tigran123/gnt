\bibpdfbookmark{Introduction}{intro}
\pagestyle{fancy}
\thispagestyle{empty}
\bibmark{book}{\scshape Introduction}
\begin{center}
\LARGE\scshape Introduction
\end{center}
\normalsize

This Second Edition of the Greek New Testament contains the text of 
five printed editions and all the readings of known Greek
manuscripts and ancient versions.
The First Edition has been published in September 2005 and
contained a full collation of four printed editions but a relatively small
number of readings from the manuscripts.
Therefore in the present work we are filling in the gap left by the
First Edition and presenting \bibemph{all} textual variants from the
manuscripts and ancient versions.

No variations, however strongly supported by the Greek manuscripts
and other witnesses, have been introduced into the text, but have been
relegated to the footnotes.
While the modern chapter and verse numbering has been preserved for the sake of
convenient reference, elements that rely on traditions and interpretations,
such as capital letters and punctuation, are completely disregarded, as they
have no support from the ancient sources.

The following five printed editions are fully collated in the footnotes:
\renewcommand{\stru}{\rule[-2mm]{0mm}{6mm}}
\renewcommand{\bbfont}{\bbbig}
\newlength{\tabwidth}
\setlength{\tabwidth}{0.8\linewidth}
\begin{longtable}{|c|p{\tabwidth}|}
\caption*{\scshape\large List of Printed Editions}\\
\endhead
\hline
\stru\src{T} & Textus Receptus, Stephanus 1550 edition.
               The text of this edition is used as a base and all the readings
               differing from it in any way are collated in the footnotes.
	       Obvious errors present in this edition have been silently
	       corrected.\\
\hline
\stru\src{K} & F.S.A. Scrivener, Textus Receptus 1881.
               This is the Greek text underlying the English Authorized
	       Version of 1611 (\bibemph{K} stands for KJV).\\
\hline
\stru\src{M} & Byzantine Majority Text edition according to Zane~C. Hodges
               and Arthur~L. Farstad.
	       We have also noted all the differences between this and
	       the Byzantine Textform according to Maurice~A. Robinson
	       and William Pierpont (2005 edition), denoting the later
	       text by \src{M[*]} where it differs from \src{M}.\\
\hline
\stru\src{V} & The base text of NA27 and UBS4 editions.
               In the places (e.g. \bibref{Heb 9:21}) where these differ,
	       we mark the reading of UBS4 text with \src{V}
	       and that of NA27 with \src{V[*]}.\\
\hline
\stru\src{N} & Greek NT text established by means of \bibemph{Bible Numerics}
               by Ivan Panin.\\
\hline
\stru\src{TMKVN} & Abbreviation for \src{T|M|K|V|N} where all five editions
                   agree and we included readings from other witnesses.\\
\hline
\stru\src{TMK} & Abbreviation for \src{T|M|K} where \bibemph{only} these
                 three out of five editions agree.\\
\hline
\end{longtable}

The total number of footnotes in this book is \totalfnt.
The total number of \bibemph{alternative} readings recorded in the
footnotes is \totalfnvar.

The following abbreviations are used for the Greek manuscripts:
% temporarily adjust font sizes to be bigger
\renewcommand{\bibssfont}{}
\renewcommand{\myfrak}{\bgfrak}
\begin{longtable}{|c|p{\tabwidth}|}
\caption*{\scshape\large List of Greek MSS}\\
\endfirsthead
\caption*{\scshape\large List of Greek MSS \upshape\normalsize(continued)}\\
\endhead
\caption*{(continued on the next page)}\\
\endfoot
\endlastfoot
\hline
\stru\src{P.Ant.2.54} & Papyrus P.~Antinoopolis 2.54.\\
\hline
\stru\src{Pap[n]} & Papyrus number `n' according to the standard enumeration.
                    All known papyri are quoted.\\
\hline
\stru\src{aleph} & (01) Codex Sinaiticus, IV century.
                   The readings of three correctors of this Codex
	           are denoted by:
	           \src{aleph[1]} (IV--VI century),
	           \src{aleph[2]} (VII century),
	           \src{aleph[c]} (XII century).
	           The notation \src{aleph[c]} is also used in the more generic
	           sense when it is difficult to establish the specific
		   corrector's hand.\\
\hline
\stru\src{A}     & (02) Codex Alexandrinus, V century.\\
\hline
\stru\src{B}     & (03) Codex Vaticanus (no book of Revelation), IV century.
              The readings of two correctors of this Codex
	      are denoted by:
	      \src{B[1]} (contemporary with \src{B}),
	      \src{B[2]} (VI--VII century).\\
\hline
\stru\src{C}     & (04) Codex Ephraemi Rescriptus, V century.
              The readings of three correctors of this Codex
	      are denoted by:
	      \src{C[1]} (contemporary with \src{C}),
	      \src{C[2]} (VI century),
	      \src{C[3]} (IX century),\\
\hline
\stru\src{D05}   & (05) Codex Bezae Cantabrigiensis (Gospels, Acts), V century;
              The readings of three correctors of this Codex
	      are denoted by:
	      \src{D[1]} (VI--VII century),
	      \src{D[2]} (IX century),
	      \src{D[c]} (XII century).
	      The readings of the Latin side of this bilingual manuscript
	      are denoted by \src{d05}.\\
\hline
\stru\src{D06}   & (06) Codex Claromontanus (Pauline Epistles), VI century.
              The readings of three correctors of this Codex
	      are denoted by:
	      \src{D[1]} (VI--VII century),
	      \src{D[2]} (IX century),
	      \src{D[c]} (XII century).\\
\hline
\stru\src{E07}   & (07) Basel (Gospels), VIII century.\\
\hline
\stru\src{E08}   & (08) Oxford Gr.~35 (Acts), VI century.\\
\hline
\stru\src{F09}   & (09) Ultrecht Ms.~1 (Acts), IX century.\\
\hline
\stru\src{F010}  & (010) Cambridge B.~XVII.~1 (Epistles), IX century.\\
\hline
\stru\src{G011}  & (011) London Harley 5684 (Gospels), IX century.\\
\hline
\stru\src{G012}  & (012) Dresden A~145b (Epistles), IX century.\\
\hline
\stru\src{H013}  & (013) Hamburg Cod.~91 (Gospels), IX century.\\
\hline
\stru\src{H014}  & (014) Modena G.~196, $\alpha$.~V.~6.3 (Acts), IX century.\\
\hline
\stru\src{H015}  & (015) Athos, VI century.\\
\hline
\stru\src{W}     & (032) Codex Washington (Gospels), IV--V century.\\
\hline
\stru\src{I}     & (016) Codex Washington (Epistles), V century.\\
\hline
\stru\src{K017}  & (017) Paris Gr.~63 (Gospels), IX century.\\
\hline
\stru\src{K018}  & (018) Moscow V.~93, S.~97 (Epistles), IX century.\\
\hline
\stru\src{L019}  & (019) Paris Gr.~62 (Gospels), VIII century.\\
\hline
\stru\src{L020}  & (020) Rome, 39 (Acts, Epistles), IX century.\\
\hline
\stru\src{M021}  & (021) Paris Gr.~48 (Gospels), IX century.\\
\hline
\stru\src{N022}  & (022) St.~Petersburg Gr.~537 (Gospels), VI century.\\
\hline
\stru\src{O023}     & (023) Paris Gr.~1286 (Matthew), VI century.\\
\hline
\stru\src{P024}     & (024) Wolfenb\"uttel 64 (Gospels), VI century.\\
\hline
\stru\src{P025}     & (025) St.~Petersburg Gr.~225 (Acts, Epistles), IX century.\\
\hline
\stru\src{Q026}     & (026) Wolfenb\"uttel 64 (Luke, John), V century.\\
\hline
\stru\src{R027}     & (027) London, Add.~17211 (Luke), VII century.\\
\hline
\stru\src{S028}     & (028) Vatican, gr.~354 (Gospels), 949 A.D.\\
\hline
\stru\src{T029}  & (029) Vatican, (Luke, John), V century.\\
\hline
\stru\src{U030}     & (030) Venice, Gr.~1,8 (1397) (Gospels), IX century.\\
\hline
\stru\src{V031}  & (031) Moscow, V.~9, S.~399 (Gospels), IX century.\\
\hline
\stru\src{X033}     & (033) M\"unchen, 2\ts{o} Cod.~ms.~30 (Gospels), X century.\\
\hline
\stru\src{Y034}     & (034) Cambridge, Mss.~6594 (Gospels), IX century.\\
\hline
\stru\src{Z035}     & (035) Dublin, Mss.~32 (Matthew), VI century.\\
\hline
\stru\src{Gamma} & (036) St.~Petersburg, Gr.~33 (Gospels), X century.\\
\hline
\stru\src{Delta} & (037) St.~Gallen, Stiftsbibl.,~48 (Gospels), IX century.\\
\hline
\stru\src{Theta} & (038) Codex Tbilisi (Gospels), IX century.\\
\hline
\stru\src{Lambda} & (039) Oxford.~Gallen, Stiftsbibl.,~48, (Luke, John) IX century.\\
\hline
\stru\src{Ksi}   & (040) Cambridge, BFBS Mss.,~213 (Luke), VI century.\\
\hline
\stru\src{Pi}    & (041) St.~Petersburg, Gr.~34 (Gospels), IX century.\\
\hline
\stru\src{Sigma} & (042) Rossano, Mus.~Diocesano, s.n. (Matthew, Mark), VI century.\\
\hline
\stru\src{Phi}   & (043) Tirana, Sttatsarchiv, Nr.~1 (Matthew, Mark), VI century.\\
\hline
\stru\src{Psi}   & (044) Athos, Lavra, B' 52 (Gospels, Acts, Epistles), IX--X century.\\
\hline
\stru\src{Omega} & (045) Athos, Dionisiu, 10(55) (Gospels), IX century.\\
\hline
\stru\src{046}   & Numbers 01--0303 preceded by zero denote the uncial manuscripts
                   in the standard enumeration.\\
\hline
\stru\src{1}     & Numbers 1--2829 denote the miniscule manuscripts in the standard
                   enumeration.\\
\hline
\stru\src{l32}  & \src{l32}--\src{l2211} denote the lectionaries in the standard enumeration.\\
\hline
\stru\src{f[1]}  & 1, 118, 131, 209, 1582 \bibemph{et al.}, according to
              the 1902 Cambridge edition of \bibemph{Codex 1 of the
	      Gospels and its Allies} by K.~Lake.\\
\hline
\stru\src{f[13]} & 13, 69, 124, 174, 230, 346, 543, 788, 826, 828, 983, 1689,
              1709 \bibemph{et al.}, \bibemph{(The Ferrar Group)}.\\
\hline
\stru\src{f[1,13]} & Whenever the witness of \src{f[1]} and \src{f[13]} agree.\\[3pt]
\hline
\stru\src{Maj[A]} & (Revelation only), large number of mss with the commentary on Revelation
                    by Andreas of Caesarea.\\
\hline
\stru\src{Maj[K]} & (Revelation only), \bibemph{Koine} Majority text proper.\\
\hline
\stru\src{Maj}   & Majority Text based on the Byzantine mss.
              In the book of Revelation this means agreement of
	      \src{Maj[A]} and \src{Maj[K]}.\\
\hline
\end{longtable}

After the list of \bibemph{named} Greek manuscripts we sometimes show
further support for a textual variant by means of the following Latin
abbreviations:\\*[3pt]
\begin{tabulary}{\linewidth}{LJ}
\src{pc}   &  \bibemph{pauci}, i.e. a few manuscripts.\\[3pt]
\src{al}   &  \bibemph{alii}, i.e. some manuscripts (more that \src{pc}),
              which differ from the Majority text \src{Maj}.\\[3pt]
\src{pm}   &  \bibemph{permulti}, i.e. large number of manuscripts,
              when \src{Maj} is divided.\\[3pt]
\src{rell} &  \bibemph{reliqui}, i.e. the rest of manuscripts
              including \src{Maj}.
	      This is normally used when I show explicitly
	      the list of those mss I personally examined but rely on
	      claims of other people to show the rest of the textual variants
	      which I had no chance to confirm personally.\\[3pt]
\end{tabulary}

In this edition we often quote the readings from the ancient versions such as
Syriac, Latin, Coptic, Armenian, Ethiopic, Georgian, Gothic and Slavonic.
The actual textual variants are not shown in their original languages but are
translated either into Koine Greek or into English, for the reader's
convenience.

The following abbreviations are used for the ancient versions:
\begin{longtable}{|c|p{\tabwidth}|}
\caption*{\scshape\large List of Ancient Versions}\\
\endfirsthead
\caption*{\scshape\large List of Ancient Versions \upshape\normalsize(continued)}\\
\endhead
\caption*{(continued on the next page)}\\
\endfoot
\endlastfoot
\hline
\stru\src{sy[s]}   & Sinaitic Syriac, IV--V century.\\
\hline
\stru\src{sy[c]}   & Curetonian Syriac, V century.\\
\hline
\stru\src{sy[h]}   & Harklean Syriac (Joseph White and Bensley editions).\\
\hline
\stru\src{sy[hmg]} & Margin readings in the Harklean Syriac.\\
\hline
\stru\src{sy[h*]}  & Alternative readings in the Harklean Syriac.\\
\hline
\stru\src{sy[p]}   & The Peshitta version of the Gospels, Acts, Pauline letters, and
                the longer epistles.\\
\hline
\stru\src{sy[ph]}  & The Philoxenian version for the shorter epistles
                (2Pe, 2--3Jo, Jud) and Revelation (Gwynn's editions).\\
\hline
\stru\src{sy}      & Used when all Syriac witnesses agree.\\
\hline
\stru\src{it}      & Old Latin witnesses (\bibemph{Itala}).\\
\hline
\stru\src{vg[mss]} & Manuscripts of Latin Vulgate.\\
\hline
\stru\src{vg[ms]}  & Single manuscript of Latin Vulgate.\\
\hline
\stru\src{vg[s]}   & Sixtine edition of Latin Vulgate, 1590.\\
\hline
\stru\src{vg[cl]}  & Clementine edition of Latin Vulgate, 1592.\\
\hline
\stru\src{vg[ww]}  & Wordsworth/White/Sparks edition of Latin Vulgate, 1889--1954.\\
\hline
\stru\src{vg[st]}  & Stuttgart edition of Latin Vulgate, 1975.\\
\hline
\stru\src{vg}      & Latin Vulgate (when all editions agree).\\
\hline
\stru\src{latt}    & Used when all Latin witnesses agree.\\
\hline
\stru\src{lat(t)}  & Used when \bibemph{almost} all Latin witnesses agree.\\
\hline
\stru\src{lat}     & Used when \src{vg} and part of \src{it} witnesses agree.\\
\hline
\stru\src{co[sa]}  & Sahidic Coptic.\\
\hline
\stru\src{co[bo]}  & Boharic Coptic.\\
\hline
\stru\src{co[pbo]} & Proto-Boharic Coptic.\\
\hline
\stru\src{co[mae]} & Middle Egyptian Coptic.\\
\hline
\stru\src{co[mf]}  & Middle Egyptian Fayyumic Coptic.\\
\hline
\stru\src{co[ac]}  & Achmimic Coptic.\\
\hline
\stru\src{co[ac2]} & Sub-Achmimic Coptic.\\
\hline
\stru\src{co[sa(ms)]} & One Sahidic manuscript witness.\\
\hline
\stru\src{co[sa(mss)]} & Two or more Sahidic manuscripts.\\
\hline
\stru\src{co[bo(ms)]} & One Boharic manuscript witness.\\
\hline
\stru\src{co[bo(mss)]} & Between two and four Boharic manuscripts.\\
\hline
\stru\src{co[bo(pt)]} & Five or more Boharic manuscripts.\\
\hline
\stru\src{co}      & Used when all Coptic witnesses agree.\\
\hline
\stru\src{arm}     & Armenian (Mesrob), V century.\\
\hline
\stru\src{eth}     & Ethiopic, VI century.\\
\hline
\stru\src{geo}     & Georgian (Blake, Bri\`ere, Garitte), V century.\\
\hline
\stru\src{got}     & Gothic (Streitberg), IV century.\\
\hline
\stru\src{sla}     & Old Church Slavonic.\\
\hline
\end{longtable}

% reset the superscript size to the small one used in the footnotes
\renewcommand{\bibssfont}{\fontsize{5}{5}\selectfont}
% reset the fraktur font size to the one used in the footnotes
\renewcommand{\myfrak}{\smfrak}
\renewcommand{\bbfont}{\bbsmall}

The word on which there is a textual variant note has a little circle \r{}.
If the textual variant affects multiple words then only the end of the
first word is marked with the circle.
If the word with a footnote occurs multiple times in a verse but all the
footnotes are identical, only the first footnote is printed but the circle
is placed on each occurrence (\bibemph{implicit footnotes}).
The \bibempty\ sign in the footnotes indicates that the text is missing.
A special symbol \bibvsend\ is used as a verse terminator.

The following symbols are used in the superscript with the manuscripts.\\*[3pt]
\newcommand{\s}[1]{\begingroup\footnotesize#1\endgroup}
\begin{longtable}{|c|p{\tabwidth}|}
\hline
\s{*}     & the \bibemph{first hand} of the manuscript (when there was an
            alteration).\\[3pt]
\hline
\s{c}     & a \bibemph{correction} of the manuscript.\\[3pt]
\hline
\s{1,2,3} & different redactions of a manuscript.\\[3pt]
\hline
\s{s}     & (= \bibemph{supplementum}) a reading of the later addition
            to a manuscript (e.g. replacing a lost section etc).\\[3pt]
\hline
\s{vl}	  & (= \bibemph{varia lectio} a reading recorded in a manuscript
            as an alternative reading.\\[3pt]
\hline
\s{txt}	  & (= \bibemph{textus} a reading recorded in the text of a manuscript
            which also records an alternative reading.\\[3pt]
\hline
\s{mg}    & (= \bibemph{in margine}) a reading in the margin of a manuscript.\\[3pt]
\hline
\s{vid}  & (= \bibemph{ut videtur}) indicates that the reading cannot be determined
            with 100\% accuracy because of the current state of the manuscript in question.\\[3pt]
\hline
\end{longtable}

If any misprints are found in this book, the editor will be glad to
be informed of them via email:
\href{mailto:info@bibles.org.uk}{info@bibles.org.uk}.
It is a great pleasure to acknowledge the helpful contributions made by the
following people (in alphabetical order by first name):
Andreas Matthias,
Anoush Yavrian,
Antonis Tsolomitis,
Apostolos Syropoulos,
Claudio Estrugo,
David Instone-Brewer,
Donald Arseneau,
Elizabeth Magba,
Eric Archibald,
Ewan MacLeod,
Heiko Oberdiek,
Jonathan Melville,
Lucy Khachoyan,
Mark Shoulson,
Piet van Oostrum,
Sebastian Rahtz,
Taras Dyatlik,
Victor Zhuromsky,
Vladimir Volovich
and
Yannis Haralambous.

And, most of all, I thank and praise the \textsc{Lord} God of Israel
for providing everything his servant needed for preserving his precious
words in this generation.
May the \textsc{Lord} use the labours of all his servants to open the
eyes of many in Israel and in all nations.

\begin{flushright}
\itshape
Tigran Aivazian\\
London, England.
\end{flushright}
