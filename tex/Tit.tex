\bibbook{H PROS TITON\\\LARGE EPISTOLH PAULOU\r{}}{PROS TITON}{Titus}{Tit}[PROS TITON]
\vs Tit 1:1
pa~uloc
\fnt{\src{T}: H PROS TITON EPISTOLH PAULOU, \src{K}: PAULOU H PROS TITON EPISTOLH, \src{M|V}: PROS TITON.}
do~uloc
vjeo~u
>ap'ostoloc
d`e
>ihso~u
qristo~u
kat`a
p'istin
>eklekt~wn
vjeo~u
ka`i
>ep'ignwsin
>alhje'iac
t~hc
kat>
e>us'ebeian\bibvsend
\vs Tit 1:2
>ep>
>elp'idi
vzw~hc
a>iwn'iou
<`hn
>ephgge'ilato
<o
>ayeud`hc
vje`oc
pr`o
qr'onwn
a>iwn'iwn\bibvsend
\vs Tit 1:3
>efan'erwsen
d`e
kairo~ic
>id'ioic
t`on
l'ogon
a>uto~u
>en
khr'ugmati
<`o
>episte'ujhn
>eg`w
kat>
>epitag`hn
to~u
swt~hroc
<hm~wn
vjeo~u\bibvsend
\vs Tit 1:4
t'itw|
gnhs'iw|
t'eknw|
kat`a
koin`hn
p'istin
q'aric
>'el\r{e}oc
\fn{\src{TMK}: eleoc, \src{V}: kai.}
e>ir'hnh
>ap`o
vjeo~u
patr`oc
ka`i
kur'iou\r{}
\fn{\src{TMK}: kuriou ihsou qristou, \src{V}: qristou ihsou.}
>ihso~u
qristo~u
to~u
swt~hroc
<hm~wn\bibvsend
\vs Tit 1:5
to'utou
q'arin
\r{k}at'elip'on
\fn{\src{TMK}: katelipon, \src{V}: apelipon.}
se
>en
kr'hth|
<'ina
t`a
le'iponta
>epidiorj'wsh|
ka`i
katast'hsh|c
kat`a
p'olin
presbut'erouc
<wc
>eg'w
soi
dietax'amhn\bibvsend
\vs Tit 1:6
e>'i
t'ic
>estin
>an'egklhtoc
mi~ac
gunaik`oc
>an`hr
t'ekna
>'eqwn
pist`a
m`h
>en
kathgor'ia|
>aswt'iac
>`h
>anup'otakta\bibvsend
\vs Tit 1:7
de~i
g`ar
t`on
>ep'iskopon
>an'egklhton
e>~inai
<wc
vjeo~u
o>ikon'omon
m`h
a>uj'adh
m`h
>org'ilon
m`h
p'aroinon
m`h
pl'hkthn
m`h
a>isqrokerd~h\bibvsend
\vs Tit 1:8
>all`a
vfil'oxenon
vfil'agajon
s'wfrona
d'ikaion
<'osion
>egkrat~h\bibvsend
\vs Tit 1:9
>anteq'omenon
to~u
kat`a
t`hn
didaq`hn
pisto~u
l'ogou
<'ina
dunat`oc
>~h|
ka`i
parakale~in
>en
t~h|
didaskal'ia|
t~h|
<ugiaino'ush|
ka`i
to`uc
>antil'egontac
>el'egqein\bibvsend
\vs Tit 1:10
e>is`in
g`ar
pollo`i
ka`i
>anup'otaktoi
mataiol'ogoi
ka`i
vfrenap'atai
m'alista
o<i
>e\r{k}
\fn{\src{TMK}: ek, \src{V}: ek thc.}
peritom~hc\bibvsend
\vs Tit 1:11
o<`uc
de~i
>epistom'izein
o<'itinec
<'olouc
o>'ikouc
>anatr'epousin
did'askontec
<`a
m`h
de~i
a>isqro~u
k'erdouc
q'arin\bibvsend
\vs Tit 1:12
e>~ip'en
tic
>ex
a>ut~wn
>'idioc
a>ut~wn
prof'hthc
kr~htec
>ae`i
ye~ustai
kak`a
vjhr'ia
gast'erec
>arga'i\bibvsend
\vs Tit 1:13
<h
martur'ia
a<'uth
>est`in
>alhj`hc
di>
<`hn
a>it'ian
>'elegqe
a>uto`uc
>apot'omwc
<'ina
<ugia'inwsin
>en
t~h|
p'istei\bibvsend
\vs Tit 1:14
m`h
pros'eqontec
>iouda"iko~ic
m'ujoic
ka`i
>entola~ic
>anjr'wpwn
>apostrefom'enwn
t`hn
>al'hjeian\bibvsend
\vs Tit 1:15
p'anta
\r{m}`en
\fn{\src{TMK}: men, \src{V}: \bibempty.}
kajar`a
to~ic
kajaro~ic
to~ic
d`e
memia\r{s}m'enoic
\fn{\src{TMK}: memiasmenoic, \src{V}: memiammenoic.}
ka`i
>ap'istoic
o>ud`en
kajar`on
>all`a
mem'iantai
a>ut~wn
ka`i
<o
no~uc
ka`i
<h
sune'idhsic\bibvsend
\vs Tit 1:16
vje`on
<omologo~usin
e>id'enai
to~ic
d`e
>'ergoic
>arno~untai
vbdelukto`i
>'ontec
ka`i
>apeije~ic
ka`i
pr`oc
p~an
>'ergon
>agaj`on
>ad'okimoi\bibvsend
\vs Tit 2:1
s`u
d`e
l'alei
<`a
pr'epei
t~h|
<ugiaino'ush|
didaskal'ia|\bibvsend
\vs Tit 2:2
presb'utac
nh\-f\r{a}\-l'i\-ouc
\fn{\src{T|K|V}: nhfal'iouc, \src{M}: nhfaleouc.}
e>~inai
semno`uc
s'wfronac
<ugia'inontac
t~h|
p'istei
t~h|
>ag'aph|
t~h|
<upomon~h|\bibvsend
\vs Tit 2:3
presb'utidac
<wsa'utwc
>en
katast'hmati
<ieroprepe~ic
m`h
diab'olouc
m`h
o>'inw|
poll~w|
dedoulwm'enac
kalodidask'alouc\bibvsend
\vs Tit 2:4
<'ina
swfron'izwsin
t`ac
n'e\-ac
vfil'androuc
e>~inai
vfilot'eknouc\bibvsend
\vs Tit 2:5
s'wfronac
<agn`ac
o>i\-kou\-r\r{o}`uc
\fn{\src{TMK}: oikourouc, \src{V}: oikourgouc.}
>agaj`ac
<upotassom'enac
to~ic
>id'ioic
>andr'asin
<'ina
m`h
<o
l'ogoc
to~u
vjeo~u
vblasfhm~htai\bibvsend
\vs Tit 2:6
to`uc
newt'erouc
<wsa'utwc
parak'alei
swfrone~in\bibvsend
\vs Tit 2:7
per`i
p'anta
seaut`on
pareq'omenoc
t'upon
kal~wn
>'ergwn
>en
t~h|
didaskal'ia|
>adiafjor'ian\r{}
\fn{\src{TMK}: adiafjorian semnothta afjarsian, \src{V}: afjorian semnothta.}
semn'othta
>afjars'ian\bibvsend
\vs Tit 2:8
l'ogon
<ugi~h
>akat'agnwston
<'ina
<o
>ex
>enant'iac
>entrap~h|
mhd`en
>'eqwn
per`i\r{}
\fn{\src{T|K}: peri umwn legein, \src{M}: peri hmwn legein, \src{V}: legein peri hmwn.}
<um~wn
l'egein
vfa~ulon\bibvsend
\vs Tit 2:9
do'ulouc
>id'ioic
desp'otaic
<upot'assesjai
>en
p~asin
e>uar'estouc
e>~inai
m`h
>antil'egontac\bibvsend
\vs Tit 2:10
m`h
nosfizom'enouc
>all`a
p'istin\r{}
\fn{\src{TMK}: pistin pasan, \src{V}: pasan pistin.}
p~asan
>endeiknum'enouc
>agaj`hn
<'ina
t`hn
didaskal'ian
t\r{o}~u
\fn{\src{TMK}: tou, \src{V}: thn tou.}
swt~hroc
<u\r{m}~wn
\fn{\src{T}: umwn, \src{K|M|V}: hmwn.}
vjeo~u
kosm~wsin
>en
p~asin\bibvsend
\vs Tit 2:11
>epef'anh
g`ar
<h
q'aric
to~u
vjeo~u
<h\r{}
\fn{\src{TMK}: h, \src{V}: \bibempty.}
swt'hrioc
p~asin
>anjr'wpoic\bibvsend
\vs Tit 2:12
paide'uousa
<hm~ac
<'ina
>arnhs'amenoi
t`hn
>as'ebeian
ka`i
t`ac
kosmik`ac
>epijum'iac
swfr'onwc
ka`i
dika'iwc
ka`i
e>useb~wc
vz'hswmen
>en
t~w|
n~un
a>i~wni\bibvsend
\vs Tit 2:13
prosdeq'omenoi
t`hn
makar'ian
>elp'ida
ka`i
>epif'aneian
t~hc
d'oxhc
to~u
meg'alou
vjeo~u
ka`i
swt~hroc
<hm~wn
>ihso~u
qristo~u\bibvsend
\vs Tit 2:14
<`oc
>'edwken
<eaut`on
<up`er
<hm~wn
<'ina
lutr'wshtai
<hm~ac
>ap`o
p'ashc
>anom'iac
ka`i
kajar'ish|
<eaut~w|
la`on
perio'usion
vzhlwt`hn
kal~wn
>'ergwn\bibvsend
\vs Tit 2:15
ta~uta
l'alei
ka`i
parak'alei
ka`i
>'elegqe
met`a
p'ashc
>epitag~hc
mhde'ic
sou
perifrone'itw\bibvsend
\vs Tit 3:1
<upom'imnhske
a>uto`uc
>arqa~ic
k\r{a}`i
\fn{\src{TMK}: kai, \src{V}: \bibempty.}
>exous'iaic
<upot'assesjai
peijarqe~in
pr`oc
p~an
>'ergon
>agaj`on
<eto'imouc
e>~inai\bibvsend
\vs Tit 3:2
mhd'ena
vblasfhme~in
>am'aqouc
e>~inai
>epieike~ic
p~asan
>endeiknum'enouc
p\r{r}a|\-'o\-th\-ta
\fn{\src{TMK}: praothta, \src{V}: prauthta.}
pr`oc
p'antac
>anjr'wpouc\bibvsend
\vs Tit 3:3
>~hmen
g'ar
pote
ka`i
<hme~ic
>an'ohtoi
>apeije~ic
plan'wmenoi
doule'uontec
>epijum'iaic
ka`i
<hdona~ic
poik'ilaic
>en
kak'ia|
ka`i
vfj'onw|
di'agontec
stughto`i
miso~untec
>all'hlouc\bibvsend
\vs Tit 3:4
<'ote
d`e
<h
qrhst'othc
ka`i
<h
vfilanjrwp'ia
>epef'anh
to~u
swt~hroc
<hm~wn
vjeo~u\bibvsend
\vs Tit 3:5
o>uk
>ex
>'ergwn
t~wn
>en
dikaios'unh|
<~w\r{n}
\fn{\src{TMK}: wn, \src{V}: a.}
>epoi'hsamen
<hme~ic
>all`a
kat`a
t`on\r{}
\fn{\src{TMK}: ton autou eleon, \src{V}: to autou eleoc.}
a>uto~u
>'eleon
>'eswsen
<hm~ac
di`a
loutro~u
paliggenes'iac
ka`i
>anakain'wsewc
pne'umatoc
<ag'iou\bibvsend
\vs Tit 3:6
o<~u
>ex'eqeen
>ef>
<hm~ac
plous'iwc
di`a
>ihso~u
qristo~u
to~u
swt~hroc
<hm~wn\bibvsend
\vs Tit 3:7
<'ina
dikaiwj'entec
t~h|
>eke'inou
q'ariti
klhron'omoi
gen'w\r{m}eja
\fn{\src{TMK}: genwmeja, \src{V}: genhjwmen.}
kat>
>elp'ida
vzw~hc
a>iwn'iou\bibvsend
\vs Tit 3:8
pist`oc
<o
l'ogoc
ka`i
per`i
to'utwn
vbo'uloma'i
se
diabebaio~usjai
<'ina
vfront'izwsin
kal~wn
>'ergwn
pro"'istasjai
o<i
pepisteuk'otec
\r{t}~w|
\fn{\src{T|K}: tw, \src{M|V}: \bibempty.}
vje~w|
ta~ut'a
>estin
\r{t}`a
\fn{\src{TMK}: ta, \src{V}: \bibempty.}
kal`a
ka`i
>wf'elima
to~ic
>anjr'wpoic\bibvsend
\vs Tit 3:9
mwr`ac
d`e
vzht'hseic
ka`i
genealog'iac
ka`i
>'ereic
ka`i
m'aqac
nomik`ac
peri"'istaso
e>is`in
g`ar
>anwfele~ic
ka`i
m'ataioi\bibvsend
\vs Tit 3:10
a<iretik`on
>'anjrwpon
met`a
m'ian
ka`i
deut'eran
noujes'ian
paraito~u\bibvsend
\vs Tit 3:11
e>id`wc
<'oti
>ex'estraptai
<o
toio~utoc
ka`i
<amart'anei
>`wn
a>utokat'akritoc\bibvsend
\vs Tit 3:12
<'otan
p'emyw
>artem~an
pr'oc
se
>`h
tuqik`on
spo'udason
>elje~in
pr'oc
me
e>ic
nik'opolin
>eke~i
g`ar
k'ekrika
paraqeim'asai\bibvsend
\vs Tit 3:13
vzhn~an
t`on
nomik`on
ka`i
>ap\r{o}ll`w
\fn{\src{TMK}: apollw, \src{V}: apollwn.}
spouda'iwc
pr'opemyon
<'ina
mhd`en
a>uto~ic
le'iph|\bibvsend
\vs Tit 3:14
manjan'etwsan
d`e
ka`i
o<i
<hm'eteroi
kal~wn
>'ergwn
pro"'istasjai
e>ic
t`ac
>anagka'iac
qre'iac
<'ina
m`h
>~wsin
>'akarpoi\bibvsend
\vs Tit 3:15
>asp'azonta'i
se
o<i
met>
>emo~u
p'antec
>'aspasai
to`uc
vfilo~untac
<hm~ac
>en
p'istei
<h
q'aric
met`a
p'antwn
<um~wn
>a\r{m}'hn\bibvsend
\fn{\src{TMK}: amhn, \src{V}: \bibempty.}
Pr`oc\r{}
\fn{\src{T}: Proc \ldots\ makedoniac, \src{K|M|V}: \bibempty.}
t'iton
t~hc
krht~wn
>ekklhs'iac
pr~wton
>ep'iskopon
qeirotonhj'enta
>egr'afh
>ap`o
nikop'olewc
t~hc
makedon'iac\bibvsend
